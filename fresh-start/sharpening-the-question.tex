\documentclass{article}
\begin{document}
\section{Sharpening the question}

My claim is that tools and results from reinforcement learning can shed light on philosophical questions concerning practical capacities.
In particular, one of the lessons to be drawn from the reinforcement learning literature is that possessing a practical capacity to do something---knowing how to do that thing---does not require very sophisticated cognitive capacities.
This conclusion is in line with so-called ``anti-intellectualist'' views.
But the battle lines in the war between intellectualists and anti-intellectualists have been unhelpfully drawn.
Before arguing that a close look at computational cognitive science lends support to (a certain brand of) anti-intellectualism, it is therefore necessary to clear some ground.
This is the purpose of the present section.

The debate about the nature of our practical capacities begins, at least in contemporary analytic philosophy, with [CITE: Ryle], and has largely revolved around the nature of know-how.
Ryle argued for two claims: first, the negative thesis that knowing how to do something is not a matter of having some (relevant) propositional attitudes, and second, the positive thesis that knowing how to do something is a matter of having a (reliable, trainable, somewhat general) disposition to do that thing.
The target of the first thesis has come to be called \emph{intellectualism}, while the latter thesis goes under the name \emph{anti-intellectualism}.\footnote{This is regrettable, since the denial of intellectualism does not entail anti-intellectualism so-defined.
Anti-intellectualism should probably have been called ``dispositionalism'' instead.
The nomenclature suggests that intellectualism and anti-intellectualism jointly exhaust the alternatives, when clearly they do not.}
%These two theses have served as attractor basins, around which a variety of related views have circled since their introduction in 1945.

The foregoing uses the slippery phrase ``is (not) a matter of.''
To gain clarity on the debate, we will need to be more precise about the content of the intellectualist and anti-intellectualist theses.
We can make some progress by distinguishing various questions to which the theses might provide answers:
\begin{enumerate}
	\item What kind of mental state (if any) is the state of knowing-how?
	\item What grounds the possession of knowledge-how?
	\item What are necessary and sufficient conditions on possessing knowledge-how?
	\item What are the semantics of know-how ascriptions?
\end{enumerate}
To a first approximation, intellectualist answers to these questions are as follows: 
\begin{enumerate}
	\item [1.i] Knowing-how to $\varphi$ is a species of propositional knowledge, consisting in knowing that such-and-such is a way to $\varphi$.
	\item [2.i] Possession of knowledge-how is grounded in having a relevant set of propositional attitudes.
	\item [3.i] Possessing a relevant set of propositional attitudes is necessary and sufficient for possessing knowledge-how.
	\item [4.i] Sentences like ``Jane knows how to bike'' are true just in case Jane has a relevant set of propositional attitudes.
\end{enumerate}


\end{document}
