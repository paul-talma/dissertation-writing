\documentclass{phil-paper}

\title{Rational Reinforcement Learning}
\author{Paul Talma}

\begin{document}
Reinforcement learning is a set of mathematical and computational tools for handling sequential decision problems.
As a branch of economics and operations research, it is closely related to the study of choice behavior and theories of rational choice in particular.
As a branch of computer science, it sits alongside supervised and unsupervised learning as one of the methodological pillars of machine machine learning.
And as a branch of  cognitive science, it stands as one of the best-confirmed computational analyses of animal behavior, unifying a wide range of observations and enjoying precisely identified neural correlates.
Despite this impressive pedigree, reinforcement learning has garnered little attention from the philosophical community.
This is a shame.
Reinforcement learning can shed light on many questions of philosophical interest, and raises interesting questions of its own.
I will indicate some of these questions below.
I claim that reinforcement learning provides an account of the structure and content of certain mental representations, and that this account reveals surprising features of these representations.
Before explaining this account in any greater detail, I will provide an overview of reinforcement learning and summarize some of the field's key findings.

% Reinforcement learning is a framework for modeling sequential decision-making.
% In addition to its use in cognitive and neural modeling [CITE: gureckis and love], reinforcement learning is one of the foundational machine learning approaches to artificial intelligence.
% According to the ``standard model,'' artificial intelligence aims to design \emph{rational agents}, understood as agents that maximize their expected utility [CITE: Russell and Norvig].
% Reinforcement learning agents do not in general maximize expected utility.
% Instead, they follow a policy learned through experience.
% In favorable circumstances, with a good learning algorithm, and with enough training, the agent learns an approximately optimal policy.
% Nonetheless, reinforcement learners fall short of full Bayesian rationality.
% Do these shortcomings merely reflect the difficulty of building an ideal Bayesian agent---engineering compromises in the face of the intractability of full Bayesian rationality?
% Or can the tools of reinforcement learning be vindicated from within the Bayesian perspective?


% \section{Bayesian Decision Theory}

The starting point for a general theory of (instrumental) rationality is Bayesian Decision Theory (BDT).
BDT is an intuitively compelling and mathematically precise theory of the interaction between an agent's credences, their utilities, and their decisions.
We do not claim that BDT is the final word concerning rationality.
Rather, we claim that it provides a helpful and promising starting point for thinking about rationality.
In the remainder of this section, we provide an overview of BDT, so as to set the backdrop for our discussion of reinforcement learning.
In brief, BDT models an agent's degrees of uncertainty over states of affairs using the probability calculus.

\subsection{Credences}
One of the central concepts of BDT is that of a \emph{credal state}.
Credal states are states that encode probability distributions.
According to \emph{probabilism}, a rational agent's credal states obey Kolmogorov's probability axioms [CITE: Titlebaum].





Credences are often explicated as \emph{degrees of belief}.
Beliefs are \emph{propositional} mental states.
BDT and reinforcement learning apply to organisms and machines that lack (or in any case are not known to possess) propositional mental states.
For example, [CITE: Rescorla Dog] analyses a dog's hunting behavior in terms BDT.
Crucially, while the dog maintains a probability distribution over mental representations, these representations lack propositional structure (they have a map-like structure).
Thus, the applicability of BDT does not depend on the agent's having propositional capacities.


% \section{Reinforcement Learning}


Reinforcement learning is a set of mathematical and computational tools for handling sequential decision problems.
As a branch of economics and operations research, it is closely related to the study of rational choice behavior and to theories of optimal control.
As a branch of computer science, it sits alongside supervised and unsupervised learning as one of the methodological pillars of machine machine learning, and lies behing fundamental advances in robotics.
And as a branch of cognitive science, it stands as one of the best-confirmed computational analyses of human and animal behavior, unifying a wide range of observations and enjoying precisely identified neural correlates.

In this section, I will provide an overview of the reinforcement learning framework and of its use in cognitive science.

\subsection{Reinforcement Learning: the Basics}

I begin with an informal gloss before introducing some technical notions.
The follwing example does not illustrate all relevant features of reinforcement learning (no single example could), but should serve as a relatively concrete scenario onto which the technical concepts introduced later can be mapped.
There will, of course, be further examples below.

Suppose that you are trapped in a cage and desirous to escape.
Around you are various levers, buttons, ropes, springs, and other bells and whistles (your captor turns out to be a certain Rube Goldberg).
For lack of an obvious way out of the cage, you haphazardly press some buttons and pull some ropes.
Nothing happens.
Then, you notice that one of the levers is connected to a latch; you press the lever before twisting a knob coupled to the latch, releasing a marble that rolls down a slide and disloges an iron bar, and you make your escape.
Unfortunately, a trapdoor opens beneath you, and after a short fall, you find yourself trapped once more in an identical cage.
This time, however, you waste no time fiddling with the buttons and ropes: you go straight for the lever and the knob, and watch as the marble secures your escape once more.
Unfortunately, an elaborate contraction of ropes, pulleys, and elastic bands catches you as you step out of the cage and transports you to yet another identical cage.
This time, curious, you decide to twist the knob without first pressing the lever.
To your surprise, this suffices to release the marble, which you now watch warily as it rolls down to unlock the door.
Careful to avoid further any further traps, you make your way out and go on to confront Prof. Goldberg.

This example illustrates several features of the reinforcement learning problem.
You are an agent, interacting with an environment.
You have a goal, which is to escape.
Achieving this goal requires performing specific sequences of actions.
Each action may have some effect on your environment.
However, the contribution of any individual action to your goal need not be clear.
Initially, you try out various actions more or less at random.
Having found, through trial and error or through careful observation of the cage mechanism, that a particular sequence of actions achieves your goal, you learn to reproduce that sequence in your next escape attempt: you do not need go through a trial-and-error process again.
However, as your third stint in the cage shows, trial and error still has a role to play.
By turning the knob without first pressing the lever, you discover that pressing the lever is not necessary to release the marble, thereby finding a more efficient route to your goal.

Abstracting from the details of this example, the core components of an reinforcement learning problem are an environment, an agent acting in that environment, potentially changing the state of and receiving feedback from the environment while working toward a goal.
Pretty clearly, if the agent is to reliably do well in its pursuit of its goal, it must also be capable of learning from its experiences.
The reinforcement learning \emph{problem} is to do well in an environment, and \emph{solutions} to this problem come in the form of algorithms for (efficiently) learning from experience within an environment (cf. [CITE: Sutton and Barto, 2].
In the remainder of this subsection, I will introduce some formalism for characterizing the problem and discuss some solutions to it.

Modern computational reinforcement learning is built upon the notion of a \emph{Markov Decision Process} (MDP).
An MDP is given by specifying a set $\mathcal S$ of states that the environment can be in, a set $\mathcal A$ of actions that the agent can undertake, a set of rewards $\mathcal R$, and a probabilistic structure dictating how rewards and next states depend on actions and current states.
A \emph{trajectory} is a discrete (temporal) sequence of states, actions, and subsequent rewards.
Moreover, we assume that $\mathcal A$ and $\mathcal R$ are at most countably infinite; in almost all cases, they are finite.
We do not, however, assume that $\mathcal X$ is finite or even countable.
Some of the most difficult and interesting problems in reinforcement learning arise in the context of vast state spaces.

In general, an agent's actions depends (perhaps probabilistically) on the current state of the environment, and the subsequent reward and environmental state depend (again probabilistically) on the agent's action.
Thus, state, action, and reward at a given time step are represented by random variables.
We represent a trajectory as follows:
\begin{align*}
	S_0, A_0, R_1, S_1, A_1, R_2, \dots
\end{align*}
That is, $S_0$ denotes the initial state of the environment, $A_0$ denotes the first action taken by the agent, $R_1$ denotes the reward received as a result of the agent's first action, $S_1$ denotes the resulting environmental state, and so on.

The environment also specifies the conditional probabilities
\begin{align*}
	P(S_{t + 1}, R_{t + 1} | S_t, A_t )
\end{align*}
which is the probability distribution over next states and rewards given the agent's action in the current state.
Crucially, the eponymous Markov property ensures that the effects of an action depend only on the current state of the environment.
That is,
\begin{align*}
	P(S_{t + 1}, R_{t + 1} | S_t, A_t, S_{t - 1}, A_{t - 1}, \dots )  = P(S_{t + 1}, R_{t + 1} | S_t, A_t )
\end{align*}
The Markov assumption enforces a kind of locality for the probabilistic (and hence causal) influence of states and actions upon one another.
Markov processes are thus memoryless.

To illustrate the foregoing, consider the classic gridworld environment [CITE: Sutton].

\begin{center}
	[insert gridworld pic here]
\end{center}

The states of the environment correspond to the cells on a grid.
In each state, the available actions correspond to the four cardinal directions.
The effect of each action is to move to the adjacent cell in the chosen direction.
The agent receives a large reward for reaching the goal state, and a small penalty for all other transitions.
These consequences of the agent's actions are captured in the transition probabilities, which in this case are deterministic (i.e. if the agent chooses to go up, they'll end up in the northern adjacent cell and get negative reward---unless the cell contains the goal---with probability $1$).
In this example, there is a clear goal state: the only state that rewards the agent for reaching it.
In other cases, rewards may be more dispersed or continuing [CITE: Abel et al. 2023].

The foregoing concludes the formal definition of an MDP.
The MDP characterizes the decision problem which the agent must solve.
Before discussing solutions to this problem, it is worth reflecting on the scope and nature of this modeling choice.
As Sutton and Barto write in their reference textbook,
\begin{quote}
	The MDP framework is abstract and flexible and can be applied to many different problems in many different ways.
	For example, the time steps need not refer to fixed intervals of real time; they can refer to arbitrary successive stages of decision making and acting.
	The actions can be low-level controls, such as the voltages applied to the motors of a robot arm, or high-level decisions, such as whether or not to have lunch or to go to graduate school.
	Similarly, the states can take a wide variety of forms.
	They can be completely determined by low-level sensations, such as direct sensor readings, or they can be more high-level and abstract, such as symbolic descriptions of objects in a room.
	%Some of what makes up a state could be based on memory of past sensations or even be entirely mental or subjective.
	%For example, an agent could be in the state of not being sure where an object is, or of having just been surprised in some clearly defined sense.
	%Similarly, some actions might be totally mental or computational.
	%For example, some actions might control what an agent chooses to think about, or where it focuses its attention.
	%In general, actions can be any decisions we want to learn how to make, and states can be anything we can know that might be useful in making them.

	\hfill 
	[CITE: Sutton and Barto, 50]
\end{quote}
%Reinforcement learning has flourished in fields as diverse as game-playing [CITE: Mnih et al. 2015], robot navigation [CITE: Tang et al. 2024], the control of plasma flows in fusion reactors [CITE: Degrave et al. 2022], and SAT-solvers [CITE: Fournier 2022].
% TODO: finish
[say something about the range of application of RL, the reward hypothesis, and the markov assumption.]

We now introduce the building blocks of the agent's solution to the sequential decision problem posed by an MDP.
An agent's actions in an environment are guided by a policy $\pi$.
A policy is a function from states and actions to probabilities: $\pi(s, a) = p$ if the probability of choosing action $a$ in state $s$ is $p$.
% TODO: link figure
\begin{center}
	INSERT GRIDWORLD WITH POLICY
\end{center}
Figure 2 displays an example of a (deterministic) policy for the gridworld (arrows determine which action the agent will take in each cell).
Other policies are obviously possible.

How should policies be evaluated?
Policies are evaluated with respect to the agent's goals.
The agent's goal is to maximize its \emph{return}.
The return is a technical notion, but it is intuitively related to utility.
It is defined as the agent's cummulative discounted reward.
Consider an agent interacting with an MDP and generating a sequence of rewards
\begin{align*}
	R_1, R_2, \dots
\end{align*}
The cummulative reward is simply the sum of these individual rewards:
\begin{align*}
	\sum_{t = 0}^\infty R_{t + 1}
\end{align*}
For various reasons, it is advisable to discount later rewards.\footnote{There are mathematical reasons to discount: doing so ensures that the cummulative discounted reward is finite, and hence mathematically tractable.
There are also ecological considerations: discounting by a fixed factor $\gamma$ is equivalent to terminating the learning episode with probability $1 - \gamma$ at each step.
Since the possibility of termination is always present in natural environments (a hawk might swoop down at any moment, or the environment might shift, so as to introduce a new learning situation) it is reasonable to build it into the framework.
Note that the undiscounted case (which can be tractable in certain cases) is a special case of the discounted case with $\gamma = 1$.
}
Thus, we consider the return:
\begin{align*}
	G := \sum_{t = 0}^\infty \gamma^t R_{t + 1}
\end{align*}
Now, the actual return depends on the actual sequence of rewards which the agent experiences.
Different choices of actions will in general lead to different sequences of reward.
The same action taken in the same state may result in a different next state, due to the stochastic nature of the environment.
And even when the same action in the same state leads to the same next state, the rewards might be different.
Thus, the actual return depends probabilistically on both the environmental dynamics and on the agent's policy (as we would expect).
To account for this dependence, we consider a policy's \emph{expected return}:\footnote{A policy and an MDP define a probability distribution over returns. Discounting is needed to ensure that this distribution has an expectation.}
\begin{align*}
	\mathbb E_\pi [G] = \mathbb E \left [ \sum_{t = 0}^\infty \gamma^t R_{t + 1} | \pi \right ]
\end{align*}
The agent's goal is to do as well as possible for itself in its environment.
Given the environment's ineliminable stochasticity, we formalize the agent's goal as that of finding a policy with maximal expected return.
And in general, we consider policies with higher expected returns to be better than ones with lower expected return.

A crucial tool in evaluating policies are \emph{value funtions}.
There are two sorts of value function: state value functions and action value functions.
In both cases, the value of an action or state is closely tied to return: it is the return that can be expected, in the long-run, when starting in that state or taking that action.
Note that since the return depends on the agent's policy, so do value functions.
We denote the value of a state $s$ under policy $\pi$ by $v_\pi(s)$ and the value of taking action $a$ in state $s$ by $q_\pi(s, a)$.
Note that the value of an action is \emph{not} the reward that results from taking that action (in a given state).
As an estimate of return, value is sensitive to long-run consequences, whereas rewards are inherently local, attaching to particular actions in particular states independently of their connection to other states.

Intuitively, if the agent's goal is to maximize expected return, then it should seek to perform high-value actions (alternatively, it should seek high-value states, but we will focus on action for simplicity).
There are two subtleties with this picture, and they drive the development of the most fundamental reinforcement learning algorithms.
The first is that action values are, at least initially, not known to the agent.\footnote{I am using ``know'' in a maximally loose sense here, for ease of exposition.}
Indeed, action values are expectations over (potentially) infinite sequences of rewards, the calculation of which is anything but trivial and requires information (such as transition probabilities between states) that may in any case not be available to the agent.
Thus, if the agent is to select actions based on their value, it must have a way of estimating these values on the basis of its experience in its environment: it needs to learn action values from experience.

Second, action values depend in general on the agent's policy.
If the agent modifies its policy, for example after learning that a given action has a high value, this change could in principle modify the value of that very action, rendering the updated policy worse than the original one.
In fact, this cannot happen: improving a policy ``locally,'' by swapping a lower-value action for a higher-value one, can only improve the one's expected return.
This result is known as the \emph{policy improvement theorem}.\footnote{A rigorous statement and proof sketch can be found in [CITE: Sutton and Barto: 78].}

Although local improvements to the policy cannot make it worse, such improvements usually change the value of other actions and states.
For example, consider the following toy environment:
\begin{center}
	INSERT SIMPLE DECISION TREE
\end{center}
Clearly, the optimal policy (in a sense to be made precise below) is to first go left in state $s_1$, then go right in state $s_2$.
Suppose that $\pi_1$ goes left in state $s_2$.
The value of going left in state $s_1$, conditional on following $\pi_1$, is low, since doing so will be followed by another left turn, and hence a low reward.
Suppose now that the agent learns that going right in state $s_2$ in fact has greater value, and so updates to a policy $\pi_2$, which is like $\pi_1$ except that it goes right in state $s_2$.
By the policy improvement theorem, $\pi_2$ is better than $\pi_1$ (as can be easily checked manually).
Notice also that the value of going left at $s_1$ is much greater under $\pi_2$ than under $\pi_1$.
Importantly, however, we have so far left it open whether the agent is sensitive to this fact.
Learning the value of going right in $s_2$ under $\pi_1$ (i.e. $q_{\pi_1}(s_2, \texttt{right})$) is quite distinct from learning the value of going left at $s_1$ under $\pi_2$ (i.e. $q_{\pi_2}(s_1, \texttt{left})$).
And until this new value is learned, the agent's estimate of action values in state $s_1$ will be mistaken, relative to the new policy $\pi_2$.

Thus, although the first remark above points to the need for learning action values (or related quantities, such as state values), the second shows that such learned values may be invalidated as soon as they are used---as they should be---to improve one's policy.
This tension suggests that whatever method we use to learn action values should not be too costly, as we might need to re-learn these values as soon as our estimates are used to improve our policy.


%These ``choice probabilities'' raise an interesting question: should we interpret them as subjective degrees of uncertainty, along Bayesian lines, or as objective propensities?\footnote{See [CITE: Luce, Hutteger] on choice propensities.}
%The reinforcement learning framework is agnostic with respect to this question, which should probably be decided on a case-by-case basis.
%At any rate, nothing we say will turn on the answer to this question.

State and action values obey a consistency condition called the \emph{Bellman recurrence}:
\begin{align*}
	v_{\pi}(s) = \sum_a \pi(s, a) \sum_{s\prm, r} p(s\prm, r | s, a) [r + \gamma v_\pi(s\prm)]
\end{align*}
This states that the value of state $s$ under policy $\pi$ is identical to the expected immediate reward associated with being in state $s$ and following $\pi$, plus the value of the resulting state $s\prm$.
Intuitively, present value is one-step reward plus subsequent value.
A similar equation holds for action values, according to which the value of a given action in a given state is the expected reward of taking that action plus the value of the expected action in the resulting state.

The Bellman equation is of fundamental importance to reinforcement learning.
It enables an iterative approach to learning values and policies that elegantly bypasses the concerns raised in the previous paragraphs.
In the next few paragraphs, we present a widely-known and illustrative algorithm that leverages the Bellman recurrence to learn an optimal policy.


TODO:
\begin{enumerate}
	\item algorithms
		\begin{enumerate}
			\item explore exploit
			\item policy iteration
			\item Q learning
		\end{enumerate}
\end{enumerate}

\subsection{Reinforcement Learning in Cognitive Science}

Reinforcement learning has a long and distinguished history in the cognitive sciences.
In the early 1980s, computer scientists and cognitive scientists observed that the then-dominant Rescorla-Wagner model of classical conditioning [CITE: Rescorla and Wagner] could be subsumed under the method of temporal differences [CITE: Sutton and Barto 1981].

The development of the temporal difference model of classical conditioning throughout the eighties met with great empirical success.
The model elegantly unified a variety of puzzling phenomena related to learning.
For example, while the Rescorla-Wagner model could account for blocking, it did not have the resources to capture higher-order conditioning.\footnotemark
\footnotetext{Blocking occurs when previously learned associations prevent the formation of new associations.
For example, suppose an animal has been trained to associate a tone with the delivery of food.
If the tone is then combined with another stimulus (such as a light) while the rest of the learning setup remains unchanged, the animal will fail to learn an association between the light and the food.
The prior tone-food association blocks learning a light-food associationl.
One of the great successes of the Rescorla-Wagner model was its elegant explanation of blocking.
Roughly, the model posits that learning occurs only when something surprising happens.
Since in blocking cases, the reward is fully predicted by the tone, its delivery is not surprising.
There is no surprise ``left over'' to fuel learning of a light-food association, and so the model correctly predicts that learning will not occur.

Higher-order conditioning occurs when an animal forms associations between two stimuli that have not been presented together.
For example, suppose that an animal is taught a tone-food association and is then repeatedly exposed to a light-tone association (without food delivery).
The animal exhibits higher-order conditioning if it learns a light-food association.
Note that the animal has never experienced any (immediate or delayed) connection between light and food.
However, it has learned that the light is predictive of a tone, which is in turn predictive of food.
(There are subtleties of experimental design that necessitate great care in setting up higher-order conditioning experiments: since a correlation between light and food would undermine the logic of the experiment, food cannot be presented during the light-tone trials.
But if tones are presented without being followed by food, the tone-food association undergoes extinction, and becomes unable to support higher-order conditioning.
Thus, the light-tone trials must be interspersed with tone-food trials.
But this interleaving now risks introducing some degree of correlation between the light and food, again jeopardizing any inference to true higher-order conditioning.
Fortunately, statistical methods can be used to confirm that animals indeed undergo higher-order conditioning.)
}
By contrast, both blocking and higher-order conditioning are easily seen to be consequences of the same prediction error mechanism at the heart of the temporal difference model.

In addition, the temporal difference model allowed for much greater temporal resolution than existing models.
Indeed, the basic unit of temporal organization in the Rescorla-Wagner model is the \emph{trial}: during a trial, the animal may be presented with any number of stimuli, separated by various intervals; the model sees learning as updating parameters from one trial to the next.
As such, it is blind to the finer temporal structure of trials, and cannot model within-trial learning.
The temporal difference model, by contrast, affords experimenters a fine-grained view into the temporal structure of a single trial.
As a result, a variety of factors that could not even be expressed in the Rescorla-Wagner model---such as the temporal distance between stimuli (the \emph{interstimulus interval}), temporal overlap and adjacency of stimuli, and various subtle manifestations of blocking---were successfully modeled [CITE: Kehoe, Schreurs, and Graham 1987, Sutton 1984, 1988, Sutton and Barto 1987, 1990].
In addition, the increased (temporal and conceptual) resolution of the temporal difference model allowed researchers to frame several novel questions (a mark of good science, according to [CITE: Laudan/Lakatos?]): how is the presence or absence of a stimulus across a period of time registered by the animal?
How are the model parameters (such as learning and decay rates) set?
And, perhaps most importantly, how is the temporal difference error at the heart of the model computed?

This last question was the focus of a burst of activity in the nineties, when researches observed that midbrain dopaminergic neural activity precisely matched the reward-prediction error associated with a given task [CITE: Montague et al. 1993, Montague et al. 1995, Montague et al. 1994, 1996, Niv 2009].
The details of this correspondence are not relevant for our purposes.
It will suffice to note that many core components of temporal difference algorithms were seen to be implemented in the brain: state- and action-value estimates, prediction errors, actor and critic structures, and so on.
Contemporary research has even found neural support for more advanced forms of reinforcement learning, such as hierarchical reinforcement learning (HRL).
HRL enriches the basic MDP setup with \emph{options}, which are temporally extended action sequences that the agent can select.
Implementing an HRL model requires tracking several prediction errors at once, on distinct time scales.
Empirical support for these relatively sophisticated error signals has been found [CITE: Botvinick et al. 2009, Botvinick 2012, Diuk et al. 2013].

[contemporary questions in neuroscientific RL research: locating the neural substrates of various components of RL algorithms (actor and critic, value functions, model-based vs model-free adjudication, information gain)]

Reinforcement learning ourgrew its behaviorist roots through the development of \emph{model-based} reinforcement learners.
As Sutton and Barto put it, a model
\begin{quote}
	is something that mimics the behavior of the environment, or more generally, that allows inferences to be made about how the environment will behave.
	For example, given a state and action, the model might predict the resultant next state and next reward.
	Models are used for \emph{planning}, by which we mean any way of deciding on a course of action by considering possible future situations before they are actually experienced.
	
	\hfill [CITE: Sutton and Barto: 7]
\end{quote}
Minimally, a model must track the environmental dynamics, and be usable in deciding what to do.
Two important classes of models are \emph{distribution models} and \emph{sampling models}.
A distribution model takes the form of an algorithm, which when given a state $s$, an action $a$, a next state $s\prm$, and a reward $r$, outputs a probability $p$ of transitioning to $s\prm$ and receiving reward $r$ upon taking action $a$ in $s$.
That is, a distribution model is a way of computing some $p(s\prm, r | s, a)$ for all relevant values of $s\prm, r, s,$ and $a$.
By contrast a sampling model is an algorithm that on input $s$ and $a$ outputs a next state $s\prm$ and reward $r$ with some probability $p$.
Intuitively, a sampling model encodes a distribution, but only implicitly: it does not make the transition probabilities available for further computation [CITE: Rescorla, "Neural Implementation of Bayesian Inference"].
The distribution encoded by a sampling model is the obvious one: it assigns probability $p$ to the transition $s, a \to s\prm, r$.
Either kind of model can be used to plan, though they necessitate different algorithms.

For a simple example, consider [Example of planning with a model].

%Intuitively, a model is a way of computing the environmental dynamics: it specifies the likely consequences of the agent's actions across all possible states.
%Although in the strict sense, a model computes the true distributions, it is useful to relax this assumption: in this weaker sense, a model is a way of computing \emph{a} probability distribution over next states and rewards, for each state and action---it is not required that these distributions be the same as the environmental distributions.
%A model, then, is any algorithm that functions to implement the environmental dynamics, even if it fails to do so.
%A learner is said to be model-based if it uses a model (in the weak sense) in order to learn how to act.
%It is important to operate with the weaker notion of model, so that agents may have imperfect models that they improve through experience.

Model-based reinforcement learning supports a kind of behavioral flexibility unavailable to model-free learners.
To see this, consider the following environment.

\begin{center}
	[INSERT 3-WAY PATH HERE]
\end{center}

This environment consists of three paths to the goal, with a reward of $1$ for reaching the goal and a reward of $-1$ for each other time step.
Thus, the agent does best by taking the shortest path (namely A) to the goal.
Now suppose that some obstruction is introduced at the end of path A, blocking the path to the goal.
Note that this obstruction also blocks the intermediate path B.
Suppose a model-based learner encounters this obstruction while traveling down path A and updates its environmental model to reflect the change.
If the agent is placed in the starting state again, it will directly go for the longest path C.
This is because, when deciding which path to take, the learner can look ahead at the consequences of taking each path.
Doing so reveals that both path A and B are now very inefficient ways to reach the goal, and that path C is now the most direct route to the goal.

By contrast, a model-free learner, having encountered the obstruction at the end of path A, would take path B if placed in the starting state.
This is because the model-free learner can only update its estimate of action values by taking these actions and observing their consequences.
Since it did not yet encounter the obstacle after choosing path B, the learner cannot update its estimate of the value of that path.
Thus, path B will now appear to be better than path A and path C.
Only once path B has been taken and found to be obstructed will the agent learn the correct value for B, and ultimately choose C.

This example illustrates that model-based learners can respond flexibly to changes in the causal dynamics of their environment.
Similar examples show that the same is true of changes in the rewards associated with outcomes.
Model-based learning is freed from the need to experience transitions in order to learn their values.
The model supports a kind of generalization, whereby information acquired through one experiential trajectory bears on decisions independent of that trajectory.


Models have been associated with cognitive maps: representations of the agent's environment whose format mirrors the spatial structure of the environment [CITE: Tolman 1948, Daw et al. 2005, Rescorla 2009, Chrisippus].
We shall examing in detail the representational credentials of model-based reinforcement learning below.
For now, we detail some uses of the distinction between model-based and model-free learning in the cognitive literature.

The model-based/model-free distinction is used to explain the distinction between \emph{habitual} and \emph{goal-directed} action [CITE: Dayan, Niv, etc.].
The habitual/goal-directed distinction is itself operationalized using the notion of \emph{outcome devaluation sensitivity}.
A type of behavior is sensitive to outcome devaluation if information about the value of the consequences of a choice influences that choice.
For example, consider the following experiment.
[CITE: who did this again?] taught rats that lever presses lead to food through standard instrumental conditioning protocol.
They then fed food to the rats freely, in an environment devoid of levers.
At the same time, the test subjects were injected with a nausea-inducing drug.
Upon returning to their original lever environment with the levers disconnected to reward, the rats pressed the lever less often (you would expect this anyway, since the lever is no longer connected to anything valuable, but as it turns out, the poisoned rats' rate of lever-pressing decreased faster than that of non-poisoned rats).

In this experiment (and many others like it [CITE: Drummond and Niv, Dolan and Dayan 2013]), the rats' behavior exhibits sensitivity to outcome devaluation: if an outcome (food) is devalued (by associating it with nausea), the rats are less likely to choose actions leading to this outcome.
But notice that at no point do the rats experience any association between lever pressing and nausea.
Lever presses are only ever followed by either pleasant experiences (food in the first phase) or neutral experiences (nothing in the third phase).
For their nauseated states to influence their lever-pressing, the rats would need to associate lever-pressing with the receipt of food, and the consumption of food with nausea.
That is, they would need a rudimentary environment model---a mental structure that tracks the transition and reward structure of their environment---and a way to use this model to bring future outcomes to bear on their current decision to press the lever.
[how to justify the "goal-directed" terminology? Honestly I'm not sure there's a clear sense in which model-based behavior is goal-directed while model-free is not; rather, they're both goal-directed, but model-based learning is immediately responsive to goal changes, while model-free learning needs to visit the environment to implement the necessary changes. Need to think more about whether the cog sci terminology is just misleading, or reflects a deeper contrast.]





%Notice that the rats at no point experienced an association between lever presses and nausea.
%Thus, Q-learning algorithms would lead the rats to continue pressing the lever, at least for a while.
%Intuitively, if the rats' learning was responsive only to past action rewards, their unpleasant experiences following eating the food should have no effect on their rates of lever pressing: that they got sick after eating the food in no way changes the fact that lever-pressing has led to good outcomes in the past.
%To connect lever presses to nausea, it seems that the rats would need to connect lever presses with food and food with nausea. 
%But this goes beyond merely keeping track of the average reward that has historically followed an action (again, this connection is not affected by the association of food with nausea).
%It requires tracking the (likely) effects of one's pulling the lever and updating one's estimate of action values on the basis of the value of their outcomes.
%That is, it requires a model, and sensitivity to changes in the value of outcomes.
%
%In such experiments, the rats' behavior is sensitive to outcome devaluation. 



% Bayesian Decision Theory provides a standard for evaluating an agent's behavior, in light of his preferences and credal states.
Reinforcement Learning provides algorithms allowing agents to learn behavioral policies on the basis of their experience.
These policies serve as solutions to sequential decision problems.
Thus, it is natural to ask, how do the solutions provided by Reinforcement Learning algorithms fare by the lights of Bayesian Decision Theory?

Hutteger [CITE] has shown that under certain conditions, a certain class of reinforcement learning algorithms is in fact Bayes optimal.
His analysis, however, is highly abstract, and it is not clear how the algorithms he considers relate to those used in contemporary reinforcement learning.
In this section, we first recapitulate Hutteger's framework and results.
We then translate them to the contemporary setting of Markov Decision Processes.
This setting clarifies the import of Hutteger's results for reinforcement learning algorithms, and highlights questions it leaves unanswered.
Finally, we address the questions which our broader framework allows us to pose.

\section{RL and know-how}

The nature of know-how has occupied a prominent place in contemporary philosophy of mind and action.
Much of the literature focuses on a question raised by [CITE: Ryle]'s criticisms of \emph{intellectualism}, the view that know-how centrally involves a kind of propositional knowledge.
\emph{Anti-intellectualists} (such as Ryle) deny this.
My goal in this section [paper?] is to demonstrate that computational reinforcement learning provides a range of tools for probing the nature of knowledge-how.
I will also argue that it provides compelling counterexamples to the intellectualist thesis.
But I think that the former contribution is more important: our practical capacities exhibit rich and variegated structures; it matters more to understand these structures on their own terms than to decide whether know-how is, always and everywhere, a form of know-that.
So, at any rate, I will argue.
Still, the debate between intellectualists and anti-intellectualists provides a nice place to start, so I will begin by laying out the two positions.
Then, I will introduce the basics of reinforcement learning, sharing technical details on a need-to-know basis.
With those basics on the table, I will illustrate some central uses of reinforcement learning in the cognitive and neural sciences, with an eye to their bearing on the intellectualist position.
After explaining why I take these cases to refute intellectualism and considering rejoinders, I go on to illustrate how computational concepts from reinforcement learning can shed light on various aspects of know-how, abilities, and skills, such as their relation to chunking and motor schemata, the distinction between habitual and goal-directed behavior, the question of automatization, and ...

\section{Intellectualism and Anti-Intellectualism}

In this section, I will endeavor to clarify the two central positions.
This is not a trivial task.
The literature has revolved around three concepts: \emph{skill}, \emph{ability}, and \emph{know-how}, and their relation to propositional knowledge.
Various authors seek to identify one or more of these phenomena.
For example, [CITE: Noe], following [CITE: Ryle], maintains that know-how and ability are one.
[CITE: Stanley and Williamson] deny this, and maintain instead that know-how is a species of propositional knowledge (which anti-intellectualists deny).
Some, like [CITE: Stanley and Krakauer], identify skill and know-how but divorce them from abilities, holding that one can know how to do something without being able to do it.
And so on.
It is therefore difficult to characterize the subject matter in a theory-neutral way (that is, in a way which wouldn't draw objections from at least one party).
As such, I will talk of know-how, skills, and abilities more or less interchangeably, but without wishing to imply anythign about their identity or distinctness. 
Let us proceed by way of examples.

Examples of the phenomena in question include my knowing how to swim, how to write a philosophy paper, how to get to the grocery store, how to play a video game, how to count to 10, and how to play the bass.
To these we can add that Alva No\"e's dog knows how to catch a Frisbee [CITE: Noe, 289] and that trained rats know how to navigate out of a maze.
(Note that even [CITE: Stanley and Williamson], the most promiment contemporary proponents of intellectualism, are happy to ascribe know-how to dogs.)

The first thing that should strike us about this collection of examples is its diversity: it is \emph{ex ante} implausible, I think, that all of these examples fall under the same mental kind.
Indeed, as I shall argue below, they do not: although there are good reasons to lump them together in ordinary talk, they are underwritten by importantly different cognitive mechanisms.
Ordinary talk is (here as elsewhere) no guide to mental organization.

Be that as it may, here is the intellectualist thesis:
\begin{quote}
	\textsc{Intellectualism:} To know how to $\varphi$ is to know, for some way $w$ of doing $\varphi$, that $w$ is a way of doing $\varphi$.
\end{quote}
($\varphi$ ranges over actions.)
Note that knowing that $w$ is a way of doing $\varphi$ is knowledge of a proposition: it entails holding a propositional attitude (belief) toward the proposition that $w$ is a way to do $\varphi$.
Many intellectualists add that the proposition in question must feature the way $w$ under a \emph{practical mode of presentation} [CITE: Stanley and Williamson, Pavese].
Practical modes of presentation are presented as a species of Fregean modes of presentation.
Modes of presentation support a fine-grained notion of mental content of the kind suitable for psychological and rational explanation [CITE: Fodor, Burge, Rescorla].
Practical modes of presentation index representational content to the exercise of practical capacities.
Although (unlike some anti-intellectualists [CITE: Noe]) we do not consider this notion irredeemably obscure, our argument will not turn on it, and so we omit it from our discussion.

\textsc{Intellectualism} is, as stated, a very strong thesis.
It implies not only that knowing how to do something requires a capacity for propositional thought, but also the possession of propositional \emph{knowledge}.
Moreover, having such knowledge is not merely necessary for possessing the relevant know how.
It is constitutive of (and hence sufficient for) know-how.
Not all who call themselves ``intellectualists'' will sign on to the thesis in its full strength ([CITE: XXXX], for example, requires mere belief rather than knowledge, and at times [CITE: Stanley and Krakauer] seem to think that propositional knowledge is merely necessary for, but perhaps not constitutive of, know-how).
But, again, our arguments below will not turn on these features of the view, so we can let them stand.

It \emph{is} important, however, to distinguish \textsc{Intellectualism} from a weaker thesis: 
\begin{quote}
	\textsc{Way-Representationalism:} knowing how to $\varphi$ requires (or consists in) having some representation, whether propositionally structured or not, of a way to do $\varphi$.
\end{quote}
This weaker thesis does not require knowers-how to be capable of propositional thought, though we may add that the way in question must be represented via a practical mode of presentation.
Although I will argue this view should also be rejected, not all considerations against \textsc{Intellectualism} will apply to \textsc{Way-Representationalism}.

Finally, we must also distinguish the two foregoing views from
\begin{quote}
	\textsc{Representationalism:} knowledge-how requires some representations or other.
\end{quote}
This view is much weaker.
I do not know whether it is true, but I will suggest some tools for assessing it below.
%Supposing that at least some know-how is underwritten by representations, however, the important questions will concern the differences between the various cognitive mechanisms underpinning 

Let me foreshadow my argument before introducing the necessary background.
Drawing on case studies from computational cognitive science, I will argue that a wide range practical capacities in humans and animals---in particular, motor control and navigation---are explained by reinforcement learning.
Although such explanations are, in principle, compatible with \textsc{Intellectualism}, in practice they are not.
As I will argue, in cases of \emph{model-based} reinforcement learning, the representations invoked to explain human and animal know-how are not propositionally structured.
Hence, this know-how does not consist in propositional knowledge.
These examples will suffice to refute \textsc{Intellectualism}.
In addition, I will argue that the representations invoked to explain this know-how are not representations of ways of doing the relevant actions.
Thus, these examples also serve to refute \textsc{Way-Representationalism}.
In a later, more speculative section, I consider the bearing of \emph{model-free} reinforcement learning explanations on \textsc{Representationalism}.
[Not sure what I'll say about that yet.]


\section{Reinforcement Learning}


Reinforcement learning is a set of mathematical and computational tools for handling sequential decision problems.
As a branch of economics and operations research, it is closely related to the study of rational choice behavior and to theories of optimal control.
As a branch of computer science, it sits alongside supervised and unsupervised learning as one of the methodological pillars of machine machine learning, and lies behing fundamental advances in robotics.
And as a branch of cognitive science, it stands as one of the best-confirmed computational analyses of human and animal behavior, unifying a wide range of observations and enjoying precisely identified neural correlates.

In this section, I will provide an overview of the reinforcement learning framework and of its use in cognitive science.

\subsection{Reinforcement Learning: the Basics}

I begin with an informal gloss before introducing some technical notions.
The follwing example does not illustrate all relevant features of reinforcement learning (no single example could), but should serve as a relatively concrete scenario onto which the technical concepts introduced later can be mapped.
There will, of course, be further examples below.

Suppose that you are trapped in a cage and desirous to escape.
Around you are various levers, buttons, ropes, springs, and other bells and whistles (your captor turns out to be a certain Rube Goldberg).
For lack of an obvious way out of the cage, you haphazardly press some buttons and pull some ropes.
Nothing happens.
Then, you notice that one of the levers is connected to a latch; you press the lever before twisting a knob coupled to the latch, releasing a marble that rolls down a slide and disloges an iron bar, and you make your escape.
Unfortunately, a trapdoor opens beneath you, and after a short fall, you find yourself trapped once more in an identical cage.
This time, however, you waste no time fiddling with the buttons and ropes: you go straight for the lever and the knob, and watch as the marble secures your escape once more.
Unfortunately, an elaborate contraction of ropes, pulleys, and elastic bands catches you as you step out of the cage and transports you to yet another identical cage.
This time, curious, you decide to twist the knob without first pressing the lever.
To your surprise, this suffices to release the marble, which you now watch warily as it rolls down to unlock the door.
Careful to avoid further any further traps, you make your way out and go on to confront Prof. Goldberg.

This example illustrates several features of the reinforcement learning problem.
You are an agent, interacting with an environment.
You have a goal, which is to escape.
Achieving this goal requires performing specific sequences of actions.
Each action may have some effect on your environment.
However, the contribution of any individual action to your goal need not be clear.
Initially, you try out various actions more or less at random.
Having found, through trial and error or through careful observation of the cage mechanism, that a particular sequence of actions achieves your goal, you learn to reproduce that sequence in your next escape attempt: you do not need go through a trial-and-error process again.
However, as your third stint in the cage shows, trial and error still has a role to play.
By turning the knob without first pressing the lever, you discover that pressing the lever is not necessary to release the marble, thereby finding a more efficient route to your goal.

Abstracting from the details of this example, the core components of an reinforcement learning problem are an environment, an agent acting in that environment, potentially changing the state of and receiving feedback from the environment while working toward a goal.
Pretty clearly, if the agent is to reliably do well in its pursuit of its goal, it must also be capable of learning from its experiences.
The reinforcement learning \emph{problem} is to do well in an environment, and \emph{solutions} to this problem come in the form of algorithms for (efficiently) learning from experience within an environment (cf. [CITE: Sutton and Barto, 2].
In the remainder of this subsection, I will introduce some formalism for characterizing the problem and discuss some solutions to it.

Modern computational reinforcement learning is built upon the notion of a \emph{Markov Decision Process} (MDP).
An MDP is given by specifying a set $\mathcal S$ of states that the environment can be in, a set $\mathcal A$ of actions that the agent can undertake, a set of rewards $\mathcal R$, and a probabilistic structure dictating how rewards and next states depend on actions and current states.
A \emph{trajectory} is a discrete (temporal) sequence of states, actions, and subsequent rewards.
Moreover, we assume that $\mathcal A$ and $\mathcal R$ are at most countably infinite; in almost all cases, they are finite.
We do not, however, assume that $\mathcal X$ is finite or even countable.
Some of the most difficult and interesting problems in reinforcement learning arise in the context of vast state spaces.

In general, an agent's actions depends (perhaps probabilistically) on the current state of the environment, and the subsequent reward and environmental state depend (again probabilistically) on the agent's action.
Thus, state, action, and reward at a given time step are represented by random variables.
We represent a trajectory as follows:
\begin{align*}
	S_0, A_0, R_1, S_1, A_1, R_2, \dots
\end{align*}
That is, $S_0$ denotes the initial state of the environment, $A_0$ denotes the first action taken by the agent, $R_1$ denotes the reward received as a result of the agent's first action, $S_1$ denotes the resulting environmental state, and so on.

The environment also specifies the conditional probabilities
\begin{align*}
	P(S_{t + 1}, R_{t + 1} | S_t, A_t )
\end{align*}
which is the probability distribution over next states and rewards given the agent's action in the current state.
Crucially, the eponymous Markov property ensures that the effects of an action depend only on the current state of the environment.
That is,
\begin{align*}
	P(S_{t + 1}, R_{t + 1} | S_t, A_t, S_{t - 1}, A_{t - 1}, \dots )  = P(S_{t + 1}, R_{t + 1} | S_t, A_t )
\end{align*}
The Markov assumption enforces a kind of locality for the probabilistic (and hence causal) influence of states and actions upon one another.
Markov processes are thus memoryless.

To illustrate the foregoing, consider the classic gridworld environment [CITE: Sutton].

\begin{center}
	[insert gridworld pic here]
\end{center}

The states of the environment correspond to the cells on a grid.
In each state, the available actions correspond to the four cardinal directions.
The effect of each action is to move to the adjacent cell in the chosen direction.
The agent receives a large reward for reaching the goal state, and a small penalty for all other transitions.
These consequences of the agent's actions are captured in the transition probabilities, which in this case are deterministic (i.e. if the agent chooses to go up, they'll end up in the northern adjacent cell and get negative reward---unless the cell contains the goal---with probability $1$).
In this example, there is a clear goal state: the only state that rewards the agent for reaching it.
In other cases, rewards may be more dispersed or continuing [CITE: Abel et al. 2023].

The foregoing concludes the formal definition of an MDP.
The MDP characterizes the decision problem which the agent must solve.
Before discussing solutions to this problem, it is worth reflecting on the scope and nature of this modeling choice.
As Sutton and Barto write in their reference textbook,
\begin{quote}
	The MDP framework is abstract and flexible and can be applied to many different problems in many different ways.
	For example, the time steps need not refer to fixed intervals of real time; they can refer to arbitrary successive stages of decision making and acting.
	The actions can be low-level controls, such as the voltages applied to the motors of a robot arm, or high-level decisions, such as whether or not to have lunch or to go to graduate school.
	Similarly, the states can take a wide variety of forms.
	They can be completely determined by low-level sensations, such as direct sensor readings, or they can be more high-level and abstract, such as symbolic descriptions of objects in a room.
	%Some of what makes up a state could be based on memory of past sensations or even be entirely mental or subjective.
	%For example, an agent could be in the state of not being sure where an object is, or of having just been surprised in some clearly defined sense.
	%Similarly, some actions might be totally mental or computational.
	%For example, some actions might control what an agent chooses to think about, or where it focuses its attention.
	%In general, actions can be any decisions we want to learn how to make, and states can be anything we can know that might be useful in making them.

	\hfill 
	[CITE: Sutton and Barto, 50]
\end{quote}
%Reinforcement learning has flourished in fields as diverse as game-playing [CITE: Mnih et al. 2015], robot navigation [CITE: Tang et al. 2024], the control of plasma flows in fusion reactors [CITE: Degrave et al. 2022], and SAT-solvers [CITE: Fournier 2022].
% TODO: finish
[say something about the range of application of RL, the reward hypothesis, and the markov assumption.]

We now introduce the building blocks of the agent's solution to the sequential decision problem posed by an MDP.
An agent's actions in an environment are guided by a policy $\pi$.
A policy is a function from states and actions to probabilities: $\pi(s, a) = p$ if the probability of choosing action $a$ in state $s$ is $p$.
% TODO: link figure
\begin{center}
	INSERT GRIDWORLD WITH POLICY
\end{center}
Figure 2 displays an example of a (deterministic) policy for the gridworld (arrows determine which action the agent will take in each cell).
Other policies are obviously possible.

How should policies be evaluated?
Policies are evaluated with respect to the agent's goals.
The agent's goal is to maximize its \emph{return}.
The return is a technical notion, but it is intuitively related to utility.
It is defined as the agent's cummulative discounted reward.
Consider an agent interacting with an MDP and generating a sequence of rewards
\begin{align*}
	R_1, R_2, \dots
\end{align*}
The cummulative reward is simply the sum of these individual rewards:
\begin{align*}
	\sum_{t = 0}^\infty R_{t + 1}
\end{align*}
For various reasons, it is advisable to discount later rewards.\footnote{There are mathematical reasons to discount: doing so ensures that the cummulative discounted reward is finite, and hence mathematically tractable.
There are also ecological considerations: discounting by a fixed factor $\gamma$ is equivalent to terminating the learning episode with probability $1 - \gamma$ at each step.
Since the possibility of termination is always present in natural environments (a hawk might swoop down at any moment, or the environment might shift, so as to introduce a new learning situation) it is reasonable to build it into the framework.
Note that the undiscounted case (which can be tractable in certain cases) is a special case of the discounted case with $\gamma = 1$.
}
Thus, we consider the return:
\begin{align*}
	G := \sum_{t = 0}^\infty \gamma^t R_{t + 1}
\end{align*}
Now, the actual return depends on the actual sequence of rewards which the agent experiences.
Different choices of actions will in general lead to different sequences of reward.
The same action taken in the same state may result in a different next state, due to the stochastic nature of the environment.
And even when the same action in the same state leads to the same next state, the rewards might be different.
Thus, the actual return depends probabilistically on both the environmental dynamics and on the agent's policy (as we would expect).
To account for this dependence, we consider a policy's \emph{expected return}:\footnote{A policy and an MDP define a probability distribution over returns. Discounting is needed to ensure that this distribution has an expectation.}
\begin{align*}
	\mathbb E_\pi [G] = \mathbb E \left [ \sum_{t = 0}^\infty \gamma^t R_{t + 1} | \pi \right ]
\end{align*}
The agent's goal is to do as well as possible for itself in its environment.
Given the environment's ineliminable stochasticity, we formalize the agent's goal as that of finding a policy with maximal expected return.
And in general, we consider policies with higher expected returns to be better than ones with lower expected return.

A crucial tool in evaluating policies are \emph{value funtions}.
There are two sorts of value function: state value functions and action value functions.
In both cases, the value of an action or state is closely tied to return: it is the return that can be expected, in the long-run, when starting in that state or taking that action.
Note that since the return depends on the agent's policy, so do value functions.
We denote the value of a state $s$ under policy $\pi$ by $v_\pi(s)$ and the value of taking action $a$ in state $s$ by $q_\pi(s, a)$.
Note that the value of an action is \emph{not} the reward that results from taking that action (in a given state).
As an estimate of return, value is sensitive to long-run consequences, whereas rewards are inherently local, attaching to particular actions in particular states independently of their connection to other states.

Intuitively, if the agent's goal is to maximize expected return, then it should seek to perform high-value actions (alternatively, it should seek high-value states, but we will focus on action for simplicity).
There are two subtleties with this picture, and they drive the development of the most fundamental reinforcement learning algorithms.
The first is that action values are, at least initially, not known to the agent.\footnote{I am using ``know'' in a maximally loose sense here, for ease of exposition.}
Indeed, action values are expectations over (potentially) infinite sequences of rewards, the calculation of which is anything but trivial and requires information (such as transition probabilities between states) that may in any case not be available to the agent.
Thus, if the agent is to select actions based on their value, it must have a way of estimating these values on the basis of its experience in its environment: it needs to learn action values from experience.

Second, action values depend in general on the agent's policy.
If the agent modifies its policy, for example after learning that a given action has a high value, this change could in principle modify the value of that very action, rendering the updated policy worse than the original one.
In fact, this cannot happen: improving a policy ``locally,'' by swapping a lower-value action for a higher-value one, can only improve the one's expected return.
This result is known as the \emph{policy improvement theorem}.\footnote{A rigorous statement and proof sketch can be found in [CITE: Sutton and Barto: 78].}

Although local improvements to the policy cannot make it worse, such improvements usually change the value of other actions and states.
For example, consider the following toy environment:
\begin{center}
	INSERT SIMPLE DECISION TREE
\end{center}
Clearly, the optimal policy (in a sense to be made precise below) is to first go left in state $s_1$, then go right in state $s_2$.
Suppose that $\pi_1$ goes left in state $s_2$.
The value of going left in state $s_1$, conditional on following $\pi_1$, is low, since doing so will be followed by another left turn, and hence a low reward.
Suppose now that the agent learns that going right in state $s_2$ in fact has greater value, and so updates to a policy $\pi_2$, which is like $\pi_1$ except that it goes right in state $s_2$.
By the policy improvement theorem, $\pi_2$ is better than $\pi_1$ (as can be easily checked manually).
Notice also that the value of going left at $s_1$ is much greater under $\pi_2$ than under $\pi_1$.
Importantly, however, we have so far left it open whether the agent is sensitive to this fact.
Learning the value of going right in $s_2$ under $\pi_1$ (i.e. $q_{\pi_1}(s_2, \texttt{right})$) is quite distinct from learning the value of going left at $s_1$ under $\pi_2$ (i.e. $q_{\pi_2}(s_1, \texttt{left})$).
And until this new value is learned, the agent's estimate of action values in state $s_1$ will be mistaken, relative to the new policy $\pi_2$.

Thus, although the first remark above points to the need for learning action values (or related quantities, such as state values), the second shows that such learned values may be invalidated as soon as they are used---as they should be---to improve one's policy.
This tension suggests that whatever method we use to learn action values should not be too costly, as we might need to re-learn these values as soon as our estimates are used to improve our policy.


%These ``choice probabilities'' raise an interesting question: should we interpret them as subjective degrees of uncertainty, along Bayesian lines, or as objective propensities?\footnote{See [CITE: Luce, Hutteger] on choice propensities.}
%The reinforcement learning framework is agnostic with respect to this question, which should probably be decided on a case-by-case basis.
%At any rate, nothing we say will turn on the answer to this question.

State and action values obey a consistency condition called the \emph{Bellman recurrence}:
\begin{align*}
	v_{\pi}(s) = \sum_a \pi(s, a) \sum_{s\prm, r} p(s\prm, r | s, a) [r + \gamma v_\pi(s\prm)]
\end{align*}
This states that the value of state $s$ under policy $\pi$ is identical to the expected immediate reward associated with being in state $s$ and following $\pi$, plus the value of the resulting state $s\prm$.
Intuitively, present value is one-step reward plus subsequent value.
A similar equation holds for action values, according to which the value of a given action in a given state is the expected reward of taking that action plus the value of the expected action in the resulting state.

The Bellman equation is of fundamental importance to reinforcement learning.
It enables an iterative approach to learning values and policies that elegantly bypasses the concerns raised in the previous paragraphs.
In the next few paragraphs, we present a widely-known and illustrative algorithm that leverages the Bellman recurrence to learn an optimal policy.


TODO:
\begin{enumerate}
	\item algorithms
		\begin{enumerate}
			\item explore exploit
			\item policy iteration
			\item Q learning
		\end{enumerate}
\end{enumerate}

\subsection{Reinforcement Learning in Cognitive Science}

Reinforcement learning has a long and distinguished history in the cognitive sciences.
In the early 1980s, computer scientists and cognitive scientists observed that the then-dominant Rescorla-Wagner model of classical conditioning [CITE: Rescorla and Wagner] could be subsumed under the method of temporal differences [CITE: Sutton and Barto 1981].

The development of the temporal difference model of classical conditioning throughout the eighties met with great empirical success.
The model elegantly unified a variety of puzzling phenomena related to learning.
For example, while the Rescorla-Wagner model could account for blocking, it did not have the resources to capture higher-order conditioning.\footnotemark
\footnotetext{Blocking occurs when previously learned associations prevent the formation of new associations.
For example, suppose an animal has been trained to associate a tone with the delivery of food.
If the tone is then combined with another stimulus (such as a light) while the rest of the learning setup remains unchanged, the animal will fail to learn an association between the light and the food.
The prior tone-food association blocks learning a light-food associationl.
One of the great successes of the Rescorla-Wagner model was its elegant explanation of blocking.
Roughly, the model posits that learning occurs only when something surprising happens.
Since in blocking cases, the reward is fully predicted by the tone, its delivery is not surprising.
There is no surprise ``left over'' to fuel learning of a light-food association, and so the model correctly predicts that learning will not occur.

Higher-order conditioning occurs when an animal forms associations between two stimuli that have not been presented together.
For example, suppose that an animal is taught a tone-food association and is then repeatedly exposed to a light-tone association (without food delivery).
The animal exhibits higher-order conditioning if it learns a light-food association.
Note that the animal has never experienced any (immediate or delayed) connection between light and food.
However, it has learned that the light is predictive of a tone, which is in turn predictive of food.
(There are subtleties of experimental design that necessitate great care in setting up higher-order conditioning experiments: since a correlation between light and food would undermine the logic of the experiment, food cannot be presented during the light-tone trials.
But if tones are presented without being followed by food, the tone-food association undergoes extinction, and becomes unable to support higher-order conditioning.
Thus, the light-tone trials must be interspersed with tone-food trials.
But this interleaving now risks introducing some degree of correlation between the light and food, again jeopardizing any inference to true higher-order conditioning.
Fortunately, statistical methods can be used to confirm that animals indeed undergo higher-order conditioning.)
}
By contrast, both blocking and higher-order conditioning are easily seen to be consequences of the same prediction error mechanism at the heart of the temporal difference model.

In addition, the temporal difference model allowed for much greater temporal resolution than existing models.
Indeed, the basic unit of temporal organization in the Rescorla-Wagner model is the \emph{trial}: during a trial, the animal may be presented with any number of stimuli, separated by various intervals; the model sees learning as updating parameters from one trial to the next.
As such, it is blind to the finer temporal structure of trials, and cannot model within-trial learning.
The temporal difference model, by contrast, affords experimenters a fine-grained view into the temporal structure of a single trial.
As a result, a variety of factors that could not even be expressed in the Rescorla-Wagner model---such as the temporal distance between stimuli (the \emph{interstimulus interval}), temporal overlap and adjacency of stimuli, and various subtle manifestations of blocking---were successfully modeled [CITE: Kehoe, Schreurs, and Graham 1987, Sutton 1984, 1988, Sutton and Barto 1987, 1990].
In addition, the increased (temporal and conceptual) resolution of the temporal difference model allowed researchers to frame several novel questions (a mark of good science, according to [CITE: Laudan/Lakatos?]): how is the presence or absence of a stimulus across a period of time registered by the animal?
How are the model parameters (such as learning and decay rates) set?
And, perhaps most importantly, how is the temporal difference error at the heart of the model computed?

This last question was the focus of a burst of activity in the nineties, when researches observed that midbrain dopaminergic neural activity precisely matched the reward-prediction error associated with a given task [CITE: Montague et al. 1993, Montague et al. 1995, Montague et al. 1994, 1996, Niv 2009].
The details of this correspondence are not relevant for our purposes.
It will suffice to note that many core components of temporal difference algorithms were seen to be implemented in the brain: state- and action-value estimates, prediction errors, actor and critic structures, and so on.
Contemporary research has even found neural support for more advanced forms of reinforcement learning, such as hierarchical reinforcement learning (HRL).
HRL enriches the basic MDP setup with \emph{options}, which are temporally extended action sequences that the agent can select.
Implementing an HRL model requires tracking several prediction errors at once, on distinct time scales.
Empirical support for these relatively sophisticated error signals has been found [CITE: Botvinick et al. 2009, Botvinick 2012, Diuk et al. 2013].

[contemporary questions in neuroscientific RL research: locating the neural substrates of various components of RL algorithms (actor and critic, value functions, model-based vs model-free adjudication, information gain)]

Reinforcement learning ourgrew its behaviorist roots through the development of \emph{model-based} reinforcement learners.
As Sutton and Barto put it, a model
\begin{quote}
	is something that mimics the behavior of the environment, or more generally, that allows inferences to be made about how the environment will behave.
	For example, given a state and action, the model might predict the resultant next state and next reward.
	Models are used for \emph{planning}, by which we mean any way of deciding on a course of action by considering possible future situations before they are actually experienced.
	
	\hfill [CITE: Sutton and Barto: 7]
\end{quote}
Minimally, a model must track the environmental dynamics, and be usable in deciding what to do.
Two important classes of models are \emph{distribution models} and \emph{sampling models}.
A distribution model takes the form of an algorithm, which when given a state $s$, an action $a$, a next state $s\prm$, and a reward $r$, outputs a probability $p$ of transitioning to $s\prm$ and receiving reward $r$ upon taking action $a$ in $s$.
That is, a distribution model is a way of computing some $p(s\prm, r | s, a)$ for all relevant values of $s\prm, r, s,$ and $a$.
By contrast a sampling model is an algorithm that on input $s$ and $a$ outputs a next state $s\prm$ and reward $r$ with some probability $p$.
Intuitively, a sampling model encodes a distribution, but only implicitly: it does not make the transition probabilities available for further computation [CITE: Rescorla, "Neural Implementation of Bayesian Inference"].
The distribution encoded by a sampling model is the obvious one: it assigns probability $p$ to the transition $s, a \to s\prm, r$.
Either kind of model can be used to plan, though they necessitate different algorithms.

For a simple example, consider [Example of planning with a model].

%Intuitively, a model is a way of computing the environmental dynamics: it specifies the likely consequences of the agent's actions across all possible states.
%Although in the strict sense, a model computes the true distributions, it is useful to relax this assumption: in this weaker sense, a model is a way of computing \emph{a} probability distribution over next states and rewards, for each state and action---it is not required that these distributions be the same as the environmental distributions.
%A model, then, is any algorithm that functions to implement the environmental dynamics, even if it fails to do so.
%A learner is said to be model-based if it uses a model (in the weak sense) in order to learn how to act.
%It is important to operate with the weaker notion of model, so that agents may have imperfect models that they improve through experience.

Model-based reinforcement learning supports a kind of behavioral flexibility unavailable to model-free learners.
To see this, consider the following environment.

\begin{center}
	[INSERT 3-WAY PATH HERE]
\end{center}

This environment consists of three paths to the goal, with a reward of $1$ for reaching the goal and a reward of $-1$ for each other time step.
Thus, the agent does best by taking the shortest path (namely A) to the goal.
Now suppose that some obstruction is introduced at the end of path A, blocking the path to the goal.
Note that this obstruction also blocks the intermediate path B.
Suppose a model-based learner encounters this obstruction while traveling down path A and updates its environmental model to reflect the change.
If the agent is placed in the starting state again, it will directly go for the longest path C.
This is because, when deciding which path to take, the learner can look ahead at the consequences of taking each path.
Doing so reveals that both path A and B are now very inefficient ways to reach the goal, and that path C is now the most direct route to the goal.

By contrast, a model-free learner, having encountered the obstruction at the end of path A, would take path B if placed in the starting state.
This is because the model-free learner can only update its estimate of action values by taking these actions and observing their consequences.
Since it did not yet encounter the obstacle after choosing path B, the learner cannot update its estimate of the value of that path.
Thus, path B will now appear to be better than path A and path C.
Only once path B has been taken and found to be obstructed will the agent learn the correct value for B, and ultimately choose C.

This example illustrates that model-based learners can respond flexibly to changes in the causal dynamics of their environment.
Similar examples show that the same is true of changes in the rewards associated with outcomes.
Model-based learning is freed from the need to experience transitions in order to learn their values.
The model supports a kind of generalization, whereby information acquired through one experiential trajectory bears on decisions independent of that trajectory.


Models have been associated with cognitive maps: representations of the agent's environment whose format mirrors the spatial structure of the environment [CITE: Tolman 1948, Daw et al. 2005, Rescorla 2009, Chrisippus].
We shall examing in detail the representational credentials of model-based reinforcement learning below.
For now, we detail some uses of the distinction between model-based and model-free learning in the cognitive literature.

The model-based/model-free distinction is used to explain the distinction between \emph{habitual} and \emph{goal-directed} action [CITE: Dayan, Niv, etc.].
The habitual/goal-directed distinction is itself operationalized using the notion of \emph{outcome devaluation sensitivity}.
A type of behavior is sensitive to outcome devaluation if information about the value of the consequences of a choice influences that choice.
For example, consider the following experiment.
[CITE: who did this again?] taught rats that lever presses lead to food through standard instrumental conditioning protocol.
They then fed food to the rats freely, in an environment devoid of levers.
At the same time, the test subjects were injected with a nausea-inducing drug.
Upon returning to their original lever environment with the levers disconnected to reward, the rats pressed the lever less often (you would expect this anyway, since the lever is no longer connected to anything valuable, but as it turns out, the poisoned rats' rate of lever-pressing decreased faster than that of non-poisoned rats).

In this experiment (and many others like it [CITE: Drummond and Niv, Dolan and Dayan 2013]), the rats' behavior exhibits sensitivity to outcome devaluation: if an outcome (food) is devalued (by associating it with nausea), the rats are less likely to choose actions leading to this outcome.
But notice that at no point do the rats experience any association between lever pressing and nausea.
Lever presses are only ever followed by either pleasant experiences (food in the first phase) or neutral experiences (nothing in the third phase).
For their nauseated states to influence their lever-pressing, the rats would need to associate lever-pressing with the receipt of food, and the consumption of food with nausea.
That is, they would need a rudimentary environment model---a mental structure that tracks the transition and reward structure of their environment---and a way to use this model to bring future outcomes to bear on their current decision to press the lever.
[how to justify the "goal-directed" terminology? Honestly I'm not sure there's a clear sense in which model-based behavior is goal-directed while model-free is not; rather, they're both goal-directed, but model-based learning is immediately responsive to goal changes, while model-free learning needs to visit the environment to implement the necessary changes. Need to think more about whether the cog sci terminology is just misleading, or reflects a deeper contrast.]





%Notice that the rats at no point experienced an association between lever presses and nausea.
%Thus, Q-learning algorithms would lead the rats to continue pressing the lever, at least for a while.
%Intuitively, if the rats' learning was responsive only to past action rewards, their unpleasant experiences following eating the food should have no effect on their rates of lever pressing: that they got sick after eating the food in no way changes the fact that lever-pressing has led to good outcomes in the past.
%To connect lever presses to nausea, it seems that the rats would need to connect lever presses with food and food with nausea. 
%But this goes beyond merely keeping track of the average reward that has historically followed an action (again, this connection is not affected by the association of food with nausea).
%It requires tracking the (likely) effects of one's pulling the lever and updating one's estimate of action values on the basis of the value of their outcomes.
%That is, it requires a model, and sensitivity to changes in the value of outcomes.
%
%In such experiments, the rats' behavior is sensitive to outcome devaluation. 


\section{RL and representation learning}

In this section, we cover some of the connections between reinforcement learning and \emph{representation learning}.
Representation learning is an interdisciplinary area of research spanning machine learning and cognitive science, focused on elucidating representational structure and the ways such structures can be learned from experience.

Representation learning is particularly relevant to reinforcement learning for two reasons.
First, applications of reinforcement learning to real-world scenarios face a scaling challenge.
As the size of the state and action spaces increases, the task of finding an optimal, or nearly optimal, policy becomes computationally intractable.
This is because in such environments, states are rarely visited more than once, and often not at all.
Moreover, when a state is visited, only one out of several possible actions is taken.
Thus, even with extensive exploration, the agent cannot sample but a tiny fraction of the problem space, and hence cannot form well-informed value estimates.

Unfortunately, most real-world applications of reinforcement learning, whether natural or artificial, face intractable problem spaces [CITE: Gershman and Daw 2017, others].
Indeed, sensory stimulations are usually continuous and high-dimensional, defining uncountably infinite state-spaces.
Likewise, at any given time, the agent faces a vast range of possible actions.
Straightforward reinforcement learning algorithms are essentially powerless in the face of this complexity.

Representation learning helps to tame this complexity by learning efficient ways of summarizing information about the problem.
In particular, representation learning can help the agent manage the size of the state space, by learning useful groupings of states.
It can also help manage the size of the action space, by 

\section{Philosophical work on RL}

Philosophers have paid scant attention to reinforcement learning.
In this section, we summarize the extant literature on the subject.
Some of the literature focuses on the implications of RL for our understanding of the (human) mind, while some use tools from the philosophy of mind and action to elucidate key concepts in RL, especially as found in machine learning.
Finally, some authors place RL in the larger context of decision theory.

\subsection{RL and value}
In a pair of papers, Julia Hass argues that the success of the reinforcement learning approach in cognitive science holds lessons for our understanding of the mind.

[CITE: Haas 2022] argues that the notions of \emph{reward} and \emph{expected value} that lie at the core of the reinforcement learning framework can be used to analyze the folk-psychological notion of \emph{desire}.
To desire something, according to Haas [CITE: Haas in prep], is to subpersonally attribute a subjective reward or expected value to that thing.
The notions of reward and expected value Hass uses are supposed to be the \emph{lingua franca} of reinforcement learning.
Thus, if Haas's thesis is correct, the reinforcement learning framework offers a powerful tool for understanding a psychological state of perennial philosophical interest.

Haas also draws on the success of reinforcement learning models of selection to argue that evaluation is fundamental to the mind.
Selection encompasses a broad range of cognitive tasks in which the mind must select from among a set of alternatives.
For example, in visual perception, the mind must decide where---and on what features---to focus its attention.
In action-selection, the mind must select one among a set of alternative actions.
[CITE: studies]
And so on.

Many such selection tasks have been analyzed through the lens of reinforcement learning.
In such models, the reinforcement learning agent must choose among several options.
In many cases, the agent learns to maximize expected return by maintaining a representation of each option's value (perhaps relative to a given state).
Options are then selected on the basis of their estimated values.

Assigning value to options is therefore a core component of many successful models of various core mental processes.
In light of the explanatory success of such models, Haas argues that a wide range of basic cognitive processes implicate valuation.

\subsection{RL and action}

In a pair of papers, Butlin has argued that reinforcement learning systems are agents.
Butlin draws on [CITE: Dretske]'s theory of action to develop an account of \emph{minimal action}.
On this minimal conception, behavior must satisfy two conditions to be an action: first, it must be selectively produced in response to certain environmental conditions, where this selectivity is the result of a learning process.
Second, the learning process must function to produce behavior that is instrumentally valuable.
That is, the instrumental value of the behavior must explain why it was learned, and this explanation must ``go through'' the learning process.

This conception of action is minimal in that it does not require any of the sophisticated agential capacities that (sometime) accompany human action, such as deliberation, planning, coordination, and so on.
As such, 
Still, since behavior can only be instrumentally valuable relative to a goal, this minimal conception requires actions to be goal-directed.
Indeed, Butlin also characterizes actions as activities subject to norms, where the relevant norms derive from goals rather than functions. 
Minimal actions are thereby set apart from merely functional behavior, such as the beating of a heart.
The latter has a function---to pump blood---but serves no goal.

Butlin contends that reinforcement learning systems satisfy the conditions on minimal agency.
It is relatively uncontroversial that they satisfy the first condition.
Butlin argues that they satisfy the second condition because the learning algorithm of a reinforcement learner functions to produce instrumentally valuable behavior.
Indeed, almost all reinforcement learning systems select actions that maximize expected value (unless they are exploring).
Actions with high expected value are expected to contribute most to the long term goal of maximizing cumulative reward.
Thus, reinforcement learning systems choose actions in light of their positive contribution to long-term goals: they act instrumentally.

Butlin's account has a number of interesting philosophical consequences.
First, it implies that supervised learning systems, in contrast to reinforcement learning systems, are not agents.
Supervised learning algorithms learn by comparing their output on a given input to the correct output.



	
\end{document}

