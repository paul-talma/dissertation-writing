\section{Introduction}

The nature of know-how has occupied a prominent place in contemporary philosophy of mind and action.
Much of the literature focuses on a question raised by [CITE: Ryle]'s criticisms of \emph{intellectualism}, the view that know-how centrally involves a kind of propositional knowledge.
\emph{Anti-intellectualists} (such as Ryle) deny this.

My goal in this paper is to demonstrate that computational reinforcement learning provides a range of tools for probing the nature of knowledge-how.
I will also argue that it provides compelling counterexamples to the intellectualist thesis.
But I think that the former contribution is more important: our practical capacities exhibit rich and variegated structures; it matters more to understand these structures on their own terms than to decide whether know-how is, always and everywhere, a form of know-that.
So, at any rate, I will argue.

Still, the debate between intellectualists and anti-intellectualists provides a nice place to start, so I will begin by laying out the two positions.
Then, I will introduce the basics of reinforcement learning, sharing technical details on a need-to-know basis.
With those basics on the table, I will illustrate some central uses of reinforcement learning in the cognitive and neural sciences, with an eye to their bearing on the intellectualist position.
After explaining why I take these cases to refute intellectualism and considering rejoinders, I go on to illustrate how computational concepts from reinforcement learning can shed light on various aspects of know-how, abilities, and skills, such as their relation to chunking and motor schemata, the distinction between habitual and goal-directed behavior, the phenomenon of automatization, and ...
