\section{RL and know-how}

Reinforcement learning provides powerful tools for thinking about many of the questions at the heart of the philosophy of skill and know-how.
In this section, we argue that a closer look at cases of model-based learning undermine the intellectualist thesis.

Our argument is as follows: 
\begin{enumerate}[(i)]
\item	Some animal behavior consititutes know-how or skilled action.
\item This behavior can be explained in terms of (model-based) reinforcement learning.
\item These explanations do not warrant the ascription of propositional representations.
\item Therefore, know-how does not require, and hence does not consist in, propositional knowledge.
\item Moreover, such representations as are needed to explain this behavior do not represent ways of performing an action.
\item Thus, know-how does not require, much less consist in, representing ways of performing a task, whether propositional or not.
\end{enumerate}
Note that (iv) contradicts \textsc{Intellectualism} while (vi) contradicts \textsc{Way-Representationalism}.

Before assessing each premise, a few points of clarification.
First, note that the first and second premises should be read as follows: for some class of animal behavior, that behavior constitutes know-how and can be explained through reinforcement learning.
They should not be read as implying that any animal behavior that constitutes know-how can be explained through reinforcement learning (this would be an untenably strong claim). 

Second, on the structure of the argument.
The significant logical steps occur in the transition from (i)--(iii) to (iv) and from (v) to (vi).
These steps are not logically valid: it is logically possible that propositional representations not be required to explain a type of behavior, which nonetheless requires the possession of propositional representations.
However, I maintain that explanatory power is the principal reason to posit structured representations [CITE: Fodor, Burge].
If we find that an animal's behavior can be explained without imputing it propositional representations, we should not attribute it propositional capacities (on the basis of that behavior).
Doing so would fly in the face of the fact that representations are explanatory posits of the cognitive sciences [CITE: Fodor 1986].
As such, they must pay their ontological keep in explantory coin.

Third, this argument is meant to be consistent with the animals in question having propositional capacities.
The point at issue is not whether, say, rats are capable of propositional thought.
Rather, the question concerns whether their knowing how to navigate a maze depends on their having propositional knowledge.
My argument is that since their navigational abilities can be explained without recourse to propositional structures, their possession of these capacities does not depend on propositional knowledge.
But this is compatible with the rats nonetheless possessing propositional knowledge.

Let us now defend our premises.
Instead of defending each of (i)--(iii) and (v) individually, I will present some case studies in some detail, and then argue that they exemplify the premises.

\subsection{Rat navigation}

TODO
