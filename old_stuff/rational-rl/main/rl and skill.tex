\section{RL and skill}


\subsection{Intellectualism}

Recent work on the philosophy of skill has tended to center on the debate between \emph{intellectualism} and its denial, \emph{anti-intellectualism}.
Intellectualism is the view that knowledge-how is reducible to---that is, simply consists in---knowledge-that.
Without prejudicing the issues, knowledge-that has as its object a proposition, whereas knowledge-how has as its object a task or an action.
Thus, I know that Paris is the capital of France, and I know how to get there from Amsterdam.
That Paris is the capital of France is a proposition; getting to Paris from Amsterdam is a task, or an action.
According to intellectualism, my knowing how to get from Amsterdam to Paris consists in my knowing certain propositions, such as that the Eurostar runs from Amsterdam to Paris, that to take the Eurostar, one must buy a ticket, and so on.
More generally, intellectualists maintain that knowing how to perform a task consists in knowing that such-and-such is a way to perform that task, for some suitable such-and-such.

Not any kind of propositional knowledge will do however.
I know that pressing piano keys in the right order, at the right time, and with the right level of force is a way of performing Bach's French Suites.
But I do not know how to perform the French Suites.
Examples of this kind have pushed intellectualists to require that the relevant propositional knowledge be \emph{practical}.
Propositional knowledge is practical, in the relevant sense, if the task in question is thought of under a \emph{practical mode of presentation}.
The nature of these practical modes of presentations has been the subject of debate.
For present purposes, it suffices to say that a task is represented under a practical mode of presentation if the task representation is suitably connected to action, and specifically to motor commands relevant to the task.

Here is Carlotta Pavese giving expression to this position:
\begin{quote}
	Knowing how to perform a task, and being skilled at performing a task, such as swimming, is a matter of knowing facts about how to perform a task under a practical representation of that task.

	\hfill [CITE: Pavese 2021]
\end{quote}

I will follow Pavese, and several others, in using ``know-how'' and ``skill'' interchangeably.

In my view, intellectualism is best understood as the view that instances of skilled action must be explained (in part) by appeal to propositional capacities.
While this construal is not strictly entailed by the above definition, it would seem to be weaker: if propositional capacities were not required to explain skilled action, then creatures lacking such capacities could nonetheless perform skilled actions.
But then their knowledge of how to perform these actions would not consist in their possession of corresponding know-that.


\subsection{RL and know-how}

In this section, I will sketch an argument against intellectualism.
In a nutshell, the argument charges intellectualists with \emph{over-intellectualization} [CITE: Michael, Tyler]: know-how and skilled action is present in animals that lack propositional representational capacities.\footnote{Or to whom, at any rate, we have no reason to attribute propositional capacities.}
Thus, know-how cannot require---and a fortiori cannot consist in---propositional knowldge.

This argument has several parts.
I will begin by examining certain cases of skilled animal action.
I will argue that such actions are indeed cases of skilled action, and that the animals indeed know how to perform the relevant tasks.
Next, I will argue that this know-how can be explained without reference to propositional capacities.
Specifically, I will argue that both the acquisition and the exercise of this know-how can be explained without appeal to propositional capacities.
This conclusion contradicts the intellectualist thesis, as defined above.
Moreover, since the animals in question plausibly lack propositional capacities entirely, the cases also contradict the apparently stronger version of the intellectualist view put forward by Pavese and others.


My argument rests on my interpretation of the cases.
To 
