\section{RL and know-how}

The nature of know-how has occupied a prominent place in contemporary philosophy of mind and action.
Much of the literature focuses on a question raised by [CITE: Ryle]'s criticisms of \emph{intellectualism}, the view that know-how centrally involves a kind of propositional knowledge.
\emph{Anti-intellectualists} (such as Ryle) deny this.
My goal in this section [paper?] is to demonstrate that computational reinforcement learning provides a range of tools for probing the nature of knowledge-how.
I will also argue that it provides compelling counterexamples to the intellectualist thesis.
But I think that the former contribution is more important: our practical capacities exhibit rich and variegated structures; it matters more to understand these structures on their own terms than to decide whether know-how is, always and everywhere, a form of know-that.
So, at any rate, I will argue.
Still, the debate between intellectualists and anti-intellectualists provides a nice place to start, so I will begin by laying out the two positions.
Then, I will introduce the basics of reinforcement learning, sharing technical details on a need-to-know basis.
With those basics on the table, I will illustrate some central uses of reinforcement learning in the cognitive and neural sciences, with an eye to their bearing on the intellectualist position.
After explaining why I take these cases to refute intellectualism and considering rejoinders, I go on to illustrate how computational concepts from reinforcement learning can shed light on various aspects of know-how, abilities, and skills, such as their relation to chunking and motor schemata, the distinction between habitual and goal-directed behavior, the question of automatization, and ...

\section{Intellectualism and Anti-Intellectualism}

In this section, I will endeavor to clarify the two central positions.
This is not a trivial task.
The literature has revolved around three concepts: \emph{skill}, \emph{ability}, and \emph{know-how}, and their relation to propositional knowledge.
Various authors seek to identify one or more of these phenomena.
For example, [CITE: Noe], following [CITE: Ryle], maintains that know-how and ability are one.
[CITE: Stanley and Williamson] deny this, and maintain instead that know-how is a species of propositional knowledge (which anti-intellectualists deny).
Some, like [CITE: Stanley and Krakauer], identify skill and know-how but divorce them from abilities, holding that one can know how to do something without being able to do it.
And so on.
It is therefore difficult to characterize the subject matter in a theory-neutral way (that is, in a way which wouldn't draw objections from at least one party).
As such, I will talk of know-how, skills, and abilities more or less interchangeably, but without wishing to imply anythign about their identity or distinctness. 
Let us proceed by way of examples.

Examples of the phenomena in question include my knowing how to swim, how to write a philosophy paper, how to get to the grocery store, how to play a video game, how to count to 10, and how to play the bass.
To these we can add that Alva No\"e's dog knows how to catch a Frisbee [CITE: Noe, 289] and that trained rats know how to navigate out of a maze.
(Note that even [CITE: Stanley and Williamson], the most promiment contemporary proponents of intellectualism, are happy to ascribe know-how to dogs.)

The first thing that should strike us about this collection of examples is its diversity: it is \emph{ex ante} implausible, I think, that all of these examples fall under the same mental kind.
Indeed, as I shall argue below, they do not: although there are good reasons to lump them together in ordinary talk, they are underwritten by importantly different cognitive mechanisms.
Ordinary talk is (here as elsewhere) no guide to mental organization.

Be that as it may, here is the intellectualist thesis:
\begin{quote}
	\textsc{Intellectualism:} To know how to $\varphi$ is to know, for some way $w$ of doing $\varphi$, that $w$ is a way of doing $\varphi$.
\end{quote}
($\varphi$ ranges over actions.)
Note that knowing that $w$ is a way of doing $\varphi$ is knowledge of a proposition: it entails holding a propositional attitude (belief) toward the proposition that $w$ is a way to do $\varphi$.
Many intellectualists add that the proposition in question must feature the way $w$ under a \emph{practical mode of presentation} [CITE: Stanley and Williamson, Pavese].
Practical modes of presentation are presented as a species of Fregean modes of presentation.
Modes of presentation support a fine-grained notion of mental content of the kind suitable for psychological and rational explanation [CITE: Fodor, Burge, Rescorla].
Practical modes of presentation index representational content to the exercise of practical capacities.
Although (unlike some anti-intellectualists [CITE: Noe]) we do not consider this notion irredeemably obscure, our argument will not turn on it, and so we omit it from our discussion.

\textsc{Intellectualism} is, as stated, a very strong thesis.
It implies not only that knowing how to do something requires a capacity for propositional thought, but also the possession of propositional \emph{knowledge}.
Moreover, having such knowledge is not merely necessary for possessing the relevant know how.
It is constitutive of (and hence sufficient for) know-how.
Not all who call themselves ``intellectualists'' will sign on to the thesis in its full strength ([CITE: XXXX], for example, requires mere belief rather than knowledge, and at times [CITE: Stanley and Krakauer] seem to think that propositional knowledge is merely necessary for, but perhaps not constitutive of, know-how).
But, again, our arguments below will not turn on these features of the view, so we can let them stand.

It \emph{is} important, however, to distinguish \textsc{Intellectualism} from a weaker thesis: 
\begin{quote}
	\textsc{Way-Representationalism:} knowing how to $\varphi$ requires (or consists in) having some representation, whether propositionally structured or not, of a way to do $\varphi$.
\end{quote}
This weaker thesis does not require knowers-how to be capable of propositional thought, though we may add that the way in question must be represented via a practical mode of presentation.
Although I will argue this view should also be rejected, not all considerations against \textsc{Intellectualism} will apply to \textsc{Way-Representationalism}.

Finally, we must also distinguish the two foregoing views from
\begin{quote}
	\textsc{Representationalism:} knowledge-how requires some representations or other.
\end{quote}
This view is much weaker.
I do not know whether it is true, but I will suggest some tools for assessing it below.
%Supposing that at least some know-how is underwritten by representations, however, the important questions will concern the differences between the various cognitive mechanisms underpinning 

Let me foreshadow my argument before introducing the necessary background.
Drawing on case studies from computational cognitive science, I will argue that a wide range practical capacities in humans and animals---in particular, motor control and navigation---are explained by reinforcement learning.
Although such explanations are, in principle, compatible with \textsc{Intellectualism}, in practice they are not.
As I will argue, in cases of \emph{model-based} reinforcement learning, the representations invoked to explain human and animal know-how are not propositionally structured.
Hence, this know-how does not consist in propositional knowledge.
These examples will suffice to refute \textsc{Intellectualism}.
In addition, I will argue that the representations invoked to explain this know-how are not representations of ways of doing the relevant actions.
Thus, these examples also serve to refute \textsc{Way-Representationalism}.
In a later, more speculative section, I consider the bearing of \emph{model-free} reinforcement learning explanations on \textsc{Representationalism}.
[Not sure what I'll say about that yet.]

