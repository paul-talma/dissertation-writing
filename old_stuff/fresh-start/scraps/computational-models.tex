What is a computational model?
Distinguish: a climate scientist gives a computational model of the local weather.
This usually means that the scientist is using computational methods to develop a model of the weather.
For example, she may be using machine learning to make predictions on the basis of cloud patterns.
Or she may be running a simulation on a powerful computer.
She is using computers to explain or predict the weather, but in doing so she is not explaining the weather as a computational process.
By contrast, a computer scientist explaining how a computer checks for primality does not usually use a computer to aid his explanation (except perhaps to demonstrate the target of explanation).
Rather, he gives a computational explanation: an explanation in terms of effective procedures carried out by the machine.
Likewise, psycholinguists have given Bayesian models of the resolution of syntactic ambiguity.
On these models, the listener computes the posterior probability of various syntactic parses, conditional on a given auditory signal.
Finally, neuroscientists have explained various experimental findings of stimulus-response psychology, such as the fact that a rabbit's blink reflex is capable of associative learning, through computational models.
Such models posit that the rabbit (or some subsystem) computes a prediction error and uses it to time the blink.




As many have pointed out, such explanations can take various forms, depending on the scientist's explanatory ambitions.
These various forms all 
