
These capacities are systemstically related.
The capacity to interpret a piano piece, for example, depends, both for its acquisition and exercise, on a capacity to play the piano.
In turn, this capacity depends on a range of lower-level capacities for, e.g., coordinated movement.

Selection is central to all of these capacities.
They all call for selecting a course of action from a range of alternatives.
To play a trill, for example, the right muscles in the right fingers must be made to move at the right velocity, at the right time, with the right cyclic behavior, and so on.
Any such combination of parameters is, typically, only one among many possible combinations.
And typically, successful parameter combinations are sparse among these possibilities: there are many more ways not to play a trill than there are of playing one.
To make matters more difficult, the parameters that lead to successful action are context-dependent: a trill on a cheap electric keyboard requires different muscle activations than one on a B\"osendorfer.
Possessing such a capacity requires a way of reliably solving this selection problem.
But there are different ways of solving the selection problem, and the differences matter.
In particular, a longstanding question concerns the kinds of mental representations involved in the possession and exercise of practical capacities (if any).
% TODO: something here is not clicking.

I will argue that reinforcement learning provides a powerful perspective from which to think about practical capacities and related notions.

\section{Knowing-how}

Ryle [CITE: Ryle] argues that practical activity does not require for its ``intelligent'' performance that the agent grasp a proposition.
Contra Ryle, Stanley and Williamson [CITE: S\&W] argue that knowing how to do something consists in knowing some related propositions---and a fortiori, requires grasping a proposition.
In particular, it requires knowing, of some way of doing an action, that it is a way of doing that action.

But what of the dogs?
Dogs know how to catch frisbees, but their grasp of propositions remains contentious. 
As many philosophers have pointed out, Stanley and Williamson's so-called \emph{intellectualism} threatens to over-intellectualize know-how, requiring more representational firepower than is available to some bona fide knowers-how.

In this section, I seek to buttress these objections to intellectualism by considering the reinforcement learning paradigm.
In a nutshell, my argument will be that reinforcement learning can explain the possession of certain practical capacities, and that these explanations do not involve anything like propositional knowledge.

\subsection{Intellectualism}

Before developing the objection, let us paint the target more carefully.
Intellectualism, in its baldest contemporary form, is the following view ($\varphi$ ranges over actions):
\begin{quote}
	\textsc{Intellectualism:} knowing how to $\varphi$ consists in knowing, of some way $w$, that $w$ is a way of doing $\varphi$; moreover, $w$ is grasped under a practical mode of presentation.
\end{quote}
[CITE: SW, Pavese]
We will be interested in a related question: whether knowing how requires propositional knowledge (that $w$ is a way to $\vaprhi$).
Note that a negative answer to this question (which I hope to provide) contradicts intellectualism: if know how consists in propositional knowldge, then it certainly requires propositional knowledge.

Intellectualism is quite strong: it requires knowers-how not only to be able to grasp propositions, but also to grasp them under a specific mode of presentation, and to know them (this was, nonetheless, Ryle's original target [CITE: Ryle]).
It will therefore be worthwhile to lay out a few weaker variants that preserve some of the spirit of the view.

The first variant maintains the requirement that some way of $\varphi$ing be represented, but drops the requirement that it be represented as part of a proposition (and hence that such proposition be grasped or known). It is also stated as a necessary condition rather than as a constitutive claim, as the necessity will be our main concern.
\begin{quote}
	\textsc{Way-representationalism:} knowing how to $\varphi$ consists in representing a way of $\varphi$ing, though this way need not be represented as part of a proposition.
\end{quote}






\section{Sharpening the question}

My claim is that tools and results from reinforcement learning can shed light on philosophical questions concerning practical capacities.
In particular, one of the lessons to be drawn from the reinforcement learning literature is that possessing a practical capacity to do something---knowing how to do that thing---does not require very sophisticated cognitive capacities.
This conclusion is in line with so-called ``anti-intellectualist'' views.
But the battle lines in the war between intellectualists and anti-intellectualists have been unhelpfully drawn.
Before arguing that a close look at computational cognitive science lends support to (a certain brand of) anti-intellectualism, it is therefore necessary to clear some ground.
This is the purpose of the present section.

The debate about the nature of our practical capacities begins, at least in contemporary analytic philosophy, with [CITE: Ryle], and has largely revolved around the nature of know-how.
Ryle argued for two claims: first, the negative thesis that knowing how to do something is not a matter of having some (relevant) propositional attitudes, and second, the positive thesis that knowing how to do something is a matter of having a (reliable, trainable, somewhat general) disposition to do that thing.
The target of the first thesis has come to be called \emph{intellectualism}, while the latter thesis goes under the name \emph{anti-intellectualism}.\footnote{This is regrettable, since the denial of intellectualism does not entail anti-intellectualism so-defined.
Anti-intellectualism should probably have been called ``dispositionalism'' instead.
The nomenclature suggests that intellectualism and anti-intellectualism jointly exhaust the alternatives, when clearly they do not.}
%These two theses have served as attractor basins, around which a variety of related views have circled since their introduction in 1945.

The foregoing uses the slippery phrase ``is (not) a matter of.''
To gain clarity on the debate, we will need to be more precise about the content of the intellectualist and anti-intellectualist theses.
We can make some progress by distinguishing various questions to which the theses might provide answers:
\begin{enumerate}
	\item What kind of mental state (if any) is the state of knowing-how?
	\item What grounds the possession of knowledge-how?
	\item What are necessary and sufficient conditions on possessing knowledge-how?
	\item What are the semantics of know-how ascriptions?
\end{enumerate}
To a first approximation, intellectualist answers to these questions are as follows: 
\begin{enumerate}
	\item [1.i] Knowing-how to $\varphi$ is a species of propositional knowledge, consisting in knowing that such-and-such is a way to $\varphi$.
	\item [2.i] Possession of knowledge-how is grounded in having a relevant set of propositional attitudes.
	\item [3.i] Possessing a relevant set of propositional attitudes is necessary and sufficient for possessing knowledge-how.
	\item [4.i] Sentences like ``Jane knows how to bike'' are true just in case Jane has a relevant set of propositional attitudes.
\end{enumerate}

