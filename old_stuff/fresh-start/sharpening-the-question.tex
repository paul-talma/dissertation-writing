\section{Sharpening the question}

In this section, we lay out some of the key positions and distinctions that structure the debate on practical capacities and know-how.

Intellectualism, as Ryle originally characterizes it [CITE: Ryle], is the view that knowing how to do something consists in the possession of some relevant propositional knowledge.
Ryle was concerned to refute the view that, for example, my knowing how to play chess consists in my knowing various propositions, such as that it is good to get one's queen out early.
Contemporary advocates of intellectualism have clarified the kind of propositional knowledge at issue, giving us the following stardard statement of intellectualism ($\varphi$ ranges over actions):
\begin{quote}
	\textsc{Intellectualism}: knowing how to $\varphi$ consists in knowing, of some way $w$ of performing actions, that $w$ is a way to $\varphi$.

	\hfill [CITE: Stanley and Williamson; Pavese]
\end{quote}
Proponents of \textsc{intellectualism} often add that the way $w$ must be represented under a \emph{practical mode of presentation}.
Practical modes of presentation are supposed to be a species of Fregean modes of presentation.
Modes of presentation support a fine-grained notion of mental content of the kind suitable for psychological and rational explanation [CITE: Fodor, Burge, Rescorla, Peacocke].
Practical modes of presentation index representational content to the exercise of practical capacities [CITE: Pavese].
They are typically invoked to explain why knowing how to $\varphi$ (usually) enables one to $\varphi$.
While we (unlike some critics of intellectualism [CITE: Noe]) do not find the notion of practical modes of presentation irredeemably obscure, they will not be our concern in what follows, and so we do not explain them any further.

[Should probably say a bit more about what a ``way'' is supposed to be here, especially since that \emph{will} become relevant later.]

\textsc{Intellectualism} is a strong view.
Requiring propositional representation for know-how is already a substantial commitment, implying as it does that minds incapable of entertaining propositions lack know-how.
But \textsc{intellectualism} goes far beyond.
It requires a propositional representation (i) that $w$ is a way of doing $\varphi$, where (ii) $w$ is grasped under a specific mode of presentation (viz. practical), and (iii) this propositional representation must amount to knowledge.
Each of (i)--(iii) strengthens the already-strong view that know-how requires propositional representation.

In addition, \textsc{intellectualism} does not merely require propositional knowledge for knowing-how.
It holds that propositional knowledge is constitutive of knowledge-how.
A constitutive claim is typically stronger than a modal one [CITE: Fine]: [example here?].

Not all who identify as intellectualists would subscribe to \textsc{intellectualism} as defined above.
Some eschew practical modes of presentation [CITE!].
Others do not require knowledge but instead (true) belief [CITE!].
Yet others [CITE: Bengson and Moffett] hold that while know-how does not consist in propositional knowledge, it requires ``reasonable mastery of the concepts in a correct and complete conception of a way of $\varphi$-ing.''
All of these views, however, maintain that propositional (or conceptual)\footnote{[Connect propositional and conceptual representation]} representations are necessary for know-how: a creature incapable of entertaining propositions cannot know how to do things.

Thus, while not all intellectualist views are committed to the claim that know-how consists in propositional knowledge, they \emph{are} committed to the necessity of propositional representations for the possession of know-how.
In fact, they are committed to a slightly stronger view, which will be my main target in what follows.
They are committed to the following:
\begin{quote}
	Propositional representations are necessary to explain the possession and exercise of knowledge-how.
\end{quote}





This strengthening highlights that the issue is not whether a knower-how must be capable of propositional thought, but whether a capacity for propositional thought plays any explanatory role with respect to know-how.
The issue is not whether 

If one wished to deny, for whatever reason, that lower animals \emph{really} possess know-how, one would be left with humans, and humans clearly do possess propositional capacities.
But the question would remain: must these propositional capacities be implicated in any meaningful way in the possession or exercise of human know-how?
That is the question I take to be at the center of the debates concerning know-how and practical capacities.

- introduce weaker views, that don't depend on propositional representation (way-representationalism etc.)
- comment on the question of whether representations are involved at all—how does this connect to the original debate?
- will probably have to say something about know-how vs practical capacities.
- debate usually poorly framed; it's not propositions vs disposition. Question is whether and if so which mental capacities explain know-how/practical capacities.
